\subsection{Anonymity and certification}
\label{sec:cert}
The permissionless nature of our public chain leads to the ability for malevolent students to create educational institutes in order to get the scores for the courses they did not attend. Moreover, the knowledge students actually get by completing the course, and the conditions upon which the course is considered completed, vary significantly between the educational institutions.

These issues currently can not be solved solely on the protocol level: they require an external source of information to determine the physical existence and the reputation of an Educator. Although we leave the public chain open for the Educators to submit their private block headers, we propose to add a separate layer of reputation and trust on top of the protocol.

We do so by introducing the Archivists -- the entities that join the network with the approval from another Archivist. Their main role is to add a trust level above just the raw protocol: they store the certificates of the Educational institutes and have the right to revoke those certificates. Furthermore, the Archivists are the entities responsible for determining the ratings of the particular Educators. The Archivists base their rating on the off-chain sources of information and gain authority for providing valid ratings and performing all the necessary compliance procedures for the Educators.

While in theory the Recruiters can query the Educators directly and initiate a data disclosure request, they are generally discouraged to do so. We expect the Recruiters to be willing to pay extra fees to the Archivists for certificate validation and ranking of the Students depending on the Educators' ratings and other factors upon request. We describe the data disclosure process in detail in section \ref{sec:DataDisclosure}
