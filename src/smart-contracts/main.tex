\documentclass[a4paper]{article}

%% Language and font encodings
\usepackage[english]{babel}
\usepackage[utf8x]{inputenc}
\usepackage[T1]{fontenc}

%% Sets page size and margins
\usepackage[a4paper,top=3cm,bottom=2cm,left=3cm,right=3cm,marginparwidth=1.75cm]{geometry}

%% Useful packages
\usepackage{amsmath}
\usepackage{graphicx}
\usepackage{longtable,tabu}
\usepackage{multirow}
\usepackage{fontawesome}
\usepackage{scrextend}
\usepackage[colorinlistoftodos]{todonotes}
\usepackage[colorlinks=true, allcolors=blue]{hyperref}

\date{
  Version 0.1
  \today
}

\begin{document}

\section*{Rationale}

For Educators and Recruters to trade securely, smart contract mechanism is required.

\section*{Requirements}

The smart-contract should allow:
\begin{itemize}
  \item Secure data trading with no trust to either side;
  \item Solving conflicts;
  \item Being extensible.
\end{itemize}

\section*{Accounts}

There are 2 kinds of accounts:
\begin{itemize}
  \item Personal: created for each client, directly belongs to that client;
  \item Robot - created by client, doesn't belong directly to that client (and anyone else); represents smart contract.
\end{itemize}

Robot account should contain, aside from token mapping:
\begin{itemize}
  \item Data storage for smart contract state;
  \item The code to control the account, compiled to Plutus Core language.
\end{itemize}

The Plutus Core language allows declaring a module with exported (public) methods.
These methods will be the public API for the account.

One can evaluate the Plutus Core code itself (not just call public methods) if account directly belongs to her.
Personal accounts don't have any persistent associated code to control them.

\section*{Scripting}

The Plutus language \cite{plutus} will be used to program Robot nodes.
Any interaction with account is done via Plutus code.

The evaluation cost is collected and summed into Fee.

To prevent account-multiplication/boolean-switch-scamming attacks, we must assign relatively high cost to such operations as:
\begin{itemize}
  \item Robot-account creation;
  \item account "data" changes.
\end{itemize}

"Sandbox run" can be performed, to check costs and programming errors.

We will call natively-implemented functions to be called from Plutus Core "NIFs".

\section*{NIFs}

NIF is an (ortogonal to each other) action to be invoked on Account.
NIF is the only source of changes in Account.

The only operation outside the NIF scope is creation of account directly belonging to the client.
That should be impossible to do with NIF for security reasons.

Any operation on the Account to be implemented should be represented as a call to NIF OR as a sequence of NIF-calls.
This will allow us to reason about security/validity of operations with more confidence and limit the access and scope of each operation.

Each NIF has an invocation cost.

\section*{Transaction}

We will cover "simple money transfer" and "data disclosure" transactions in this section.

"Simple money transfer" transactions will be implemented as transactions with empty "function call" field.

Transferral transaction must contain:
\begin{itemize}
  \item Sender signature;
  \item Receiver signature;
  \item Amount and kind of funds transferred (must be non-zero overall);
  \item "Nonce" to prevent double-invocation.
  \item (Optional) textual message for the receiver;
  \item (Optional) function call to be invoked upon transaction approval.
\end{itemize}

Transaction is the only way for accounts to interact.

Function call (if present) will contain the name of exported function and arguments to be supplied upon invocation.

The function would be invoked like that:

\begin{verbatim}
function-name(Seller-signature, amount, value1, value2, ..., valueN)
\end{verbatim}

On successful code invocation, money will be transmitted to the Target account and the costs will be demanded from the Sender.
If code fails, the transaction is not published as successful and is rejected.

If there is not enough money supplied for the operation or the code raised an error, whole transaction will fail.

If there is no function call in transaction, the code invocation is assumed successful.

The fee for transaction approval will depend on NIFs invoked and will have cost calculated as \verb|NifInvoked + C * sqr(NifInvoked)| where \verb|C| is small enough so fast-ending transactions will not notice it; in the same time big transactions (like the one trying to evaluate \verb|akkermann(10, 10)|) will run out of gas.

\section*{Smart-contract mechanics}

We assume that we have 2 sides:
\begin{itemize}
  \item Buyer
  \item Seller.
\end{itemize}

"Gas" below is the estimation of the operation cost. The name and idea is taken from Etherium \cite{eth}.

Smart contracts would work as follows. Buyer invokes a transaction which runs code directly on his account, that constructs a robot with following exported methods

\begin{itemize}
  \item \begin{verbatim} initiate(); \end{verbatim}
  \item \begin{verbatim} accept-fee(); \end{verbatim}
  \item \begin{verbatim} accept(); \end{verbatim}
  \item \begin{verbatim} reject(block, proof); \end{verbatim}
  \item \begin{verbatim} refuse(); \end{verbatim}
  \item \begin{verbatim} check-time(), \end{verbatim}
\end{itemize}
carrying \verb|Sum + Gas| amount of currency and some \verb|predicate| to check the data.
\newline
\newline
Here is the state machine of that Robot-account:

\begin{verbatim}
  (0) Account was created.
    "initiate" from Buyer:
      Send an invitation to buyer
      AND GOTO 1

    "reject" from Buyer: cleanup if something is wrong
      GOTO 4

  (1) The trade was started.
        "accept" from Buyer
    AND "accept" from Seller: both accepted, terminating contract
      GOTO 3

    "reject" from Buyer
      GOTO 2

    "refuse" from Seller: in case smart contract doesn't hold enough currency.
      GOTO 4

    "check-time" from Seller
      when time-spent > TIMEOUT
        GOTO 3

    "check-time" from Buyer
      when time-spent > TIMEOUT
      AND  Seller did not respond in time
        GOTO 4

  (2) Arbitration.
    The robot invokes `check-block-and-proof` function.
    If it signals that Buyer is right (block invalidates the proof OR the predicate fails),
      GOTO 4
    else if Seller is right
      GOTO 3

  (3)
    Sum is sent to Seller.
    GOTO 4.

  (4) Cleanup. The account is closed and all remaining money
     are sent to the Buyer.

\end{verbatim}

\section*{Example}

Lets assume there are:
\begin{itemize}
  \item Seller which has declared that he has his students' Linear Algebra marks for Nov, 2018 worth 500 tokens (signed in some private Merkle tree);
  \item Buyer which has 600 tokens available.
\end{itemize}

We will consider three cases:
\begin{itemize}
  \item Seller tried to send Buyer garbage instead of data;
  \item Buyer tried to blame Seller in giving her invalid data with data being completely valid (in terms of signature and predicate).
  \item Buyer decided to not \verb|accept()| the contract.
\end{itemize}

All trades will have same initial part, so we will branch when nessessary.
The trades will go as follows:
\begin{itemize}
  \item Buyer formulates a predicate to check that data corresponds its description.

  \item She estimates the gas cost to be covered by 20 tokens max.

  \item Then she creates smart account using the scheme above with 500 + 20 tokens and the predicate.

  \item She checks that everything is right and invokes \verb|initiate()|, which notifies Seller.

  \item Seller accepts and sends Buyer encrypted data via prvate channel, along with key.

  \begin{enumerate}
    \item Seller tries to send garbage:
      \begin{itemize}
        \item Buyer decrypts the data and finds that one block signature is invalid.

        \item Then she invokes \verb|reject(unencrypted-block, proof)| to start arbitration.

        \item The smart account performs \verb|check-block-and-proof(block, proof)| and finds that Buyer was right.

        \item Then it returns all remaining funds to Buyer.
      \end{itemize}
    \item Buyer tries to blame Seller with valid data:
      \begin{itemize}
        \item Buyer selects the block to call "invalid".

        \item Then she invokes \verb|reject(unencrypted-block, proof)| to start arbitration.

        \item The smart account performs \verb|check-block-and-proof(block, proof)| and finds that Buyer was not right.

        \item Then it send 500 tokens to Seller and all remaining currency to Buyer.
      \end{itemize}
    \item Buyer receives the data, but remains silent:
      \begin{itemize}
        \item Seller after some time calls \verb|check-time()|.

        \item If the timeout is expired, 500 tokens are sent to Seller.

        \item All remaining currency is sent to Buyer.
      \end{itemize}
  \end{enumerate}

  \end{itemize}

  \begin{thebibliography}{50}
    \bibitem{plutus} \url{https://github.com/kframework/plutus-core-semantics/blob/master/docs/Plutus%20Core%20Grammar.pdf}.
    \bibitem{eth} \url{http://ethdocs.org/en/latest/contracts-and-transactions/account-types-gas-and-transactions.html}.
  \end{thebibliography}
\end{document}
